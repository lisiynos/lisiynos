\begin{section}[Простейшие методы решения]{Простейшие методы решения}
Рассмотрим, как можно решать эту задачу, не используя различные структуры данных.
\begin{itemize}

\item Наиболее простым в идее и реализации является линейный подсчет суммы чиел в массиве
при запросе $get$ за время $O(N)$ и изменение за $O(1)$.

\item Также можно предподсчитать суммы для всех возможных отрезков в матрицу $sum$ 
и запрос $get$ обрабатывать за $O(1)$, непосредственно обращаясь к нужному элементу 
матрицы. Однако при изменении числа изменяется сумма на порядка $N^2$ отрезках.
Соответственно, для корректной работы необходимо их все обновить, что требует $O(N^2)$
времени.

\item Значительно более полезным на практике является метод <<частичных сумм>>.
Заключается он в следующем: для каждой позиции $i$, $1 \le i \le N$, вычисляется 
$s(i)$~--- сумма элементов от $1$ до $i$-го включительно. Тогда запрос $get$ можно
по-прежнему обрабатывать за $O(1)$: $get(l, r) = s(r) - s(l - 1)$, при этом $s(0)$ 
считается равной $0$. Однако запрос изменения числа по-прежнему обрабатывается
долго: каждое число может участвовать в $O(N)$ частичных суммах, поэтому 
запрос изменения числа требует $O(N)$ времени на выполнение.

\end{itemize}

\end{section}