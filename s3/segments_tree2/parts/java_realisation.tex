\begin{section}[Пример реализации дерева отрезков на Java]{Пример реализации дерева отрезков на Java}

Пример реализации дерева отрезков на сумме без групповых операций.
В конструкторе $SegTree$ параметр $numberOfValues$~--- количество элементов 
в массиве, над которым строится дерево отрезков. Элементы массива читаются
из входного потока методом $nextLong$. В массиве $tree$ хранится само дерево отрезков,
$size$~--- количество листьев в дереве отрезков, $inf$~--- нейтральный элемент,
для сложения равный нулю.

Реализация снизу:

\begin{lstlisting}

public class SegTree {
    private long[] tree;
    private int size;
    private final long inf = 0;

    private void update(int vertex) {
      this.tree[vertex] = calc(this.tree[vertex * 2],
          this.tree[vertex * 2 + 1]);
    }

    private long calc(long a, long b) {
      return a + b;
    }

    public SegTree(int numberOfValues) throws IOException {
      this.size = 1;
      while (this.size < numberOfValues) {
        this.size *= 2;
      }
      this.tree = new long[this.size * 2];
      Arrays.fill(tree, inf);
      for (int i = 0; i < numberOfValues; i++)
        tree[i + size] = nextLong();
      for (int i = size - 1; i > 0; i--)
        update(i);
    }

    public void set(int pos, long newValue) {
      pos += size;
      this.tree[pos] = newValue;
      while (pos > 1) {
        pos /= 2;
        update(pos);
      }
    }

    public long get(int left, int right) {
      left += this.size;
      right += this.size;
      long answer = this.inf;
      while (left <= right) {
        if (left % 2 == 1) {
          answer = calc(answer, this.tree[left]);
          left++;
        }
        if (right % 2 == 0) {
          answer = calc(answer, this.tree[right]);
          right--;
        }
        left /= 2;
        right /= 2;
      }
      return answer;
    }
}

\end{lstlisting}

\newpage
Реализация сверху:

\begin{lstlisting}
  public class SegTree {
    private Vertex[] tree;
    private int size;
    private final long inf = 0;

    private class Vertex {
      private long left;
      private long right;
      private long value;

      public Vertex(long left, long right, long value) {
        this.left = left;
        this.right = right;
        this.value = value;
      }
    }

    public SegTree(int numberOfElements) throws IOException {
      size = 1;
      while (size < numberOfElements)
        size *= 2;
      tree = new Vertex[size * 2];
      for (int i = 0; i < numberOfElements; i++) {
        tree[i + size] = new Vertex(i, i, nextLong());
      }
      for (int i = numberOfElements; i < size; i++)
        tree[i + size] = new Vertex(i, i, inf);
      for (int i = size - 1; i > 0; i--) {
        tree[i] = new Vertex(tree[i * 2].left, tree[i * 2 + 1].right,
            tree[i * 2].value + tree[i * 2 + 1].value);
      }
    }

    public long get(int vertex, int left, int right) {
      if (tree[vertex].left > right || tree[vertex].right < left)
        return inf;
      if (tree[vertex].left >= left && tree[vertex].right <= right)
        return tree[vertex].value;
      long answer = get(vertex * 2, left, right)
          + get(vertex * 2 + 1, left, right);
      return answer;
    }

    public void set(int vertex, int left, int right, long value) {
      if (tree[vertex].left > right || tree[vertex].right < left)
        return;
      if (tree[vertex].left >= left && tree[vertex].right <= right) {
        tree[vertex].value = value;
        return;
      }
      set(vertex * 2, left, right, value);
      set(vertex * 2 + 1, left, right, value);
      tree[vertex].value = tree[vertex * 2].value
          + tree[vertex * 2 + 1].value;
    }

  }


\end{lstlisting}

\end{section}