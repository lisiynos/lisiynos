\begin{section}[Описание структуры]{Описание структуры}

Основная идея дерева отрезков такова: предположим, что длина массива является степенью
двойки (при необходимости дополним массив нулями). Для каждой степени двойки $K$, не превышащюей $N$
разобьем весь массив на последовательные отрезки длины $K$ и для каждого предподсчитаем
сумму чисел на нем. Запрос $get(l, r)$ тогда можно обрабатывать следующим образом: разбить 
отрезок с $l$ по $r$ на $O(\log(r - l))$ непересекающихся отрезков, длины которых являются
степенями двойки и сумма значений на которых уже посчитана и в силу аддитивности суммы по
отрезку сложить. Запрос $set(i, x)$ нужно будет обрабатывать следующим образом: необходимо 
увеличить значение всех $\log(N)$ отрезков, в которых лежит $i$-й элемент, на $x - v$, где
$v$~--- текущее значение $i$-го элемента.

\end{section}